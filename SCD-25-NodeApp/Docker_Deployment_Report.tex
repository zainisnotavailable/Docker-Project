\documentclass[12pt,a4paper]{article}

% Packages
\usepackage[utf8]{inputenc}
\usepackage[T1]{fontenc}
\usepackage{amsmath}
\usepackage{amssymb}
\usepackage{graphicx}
\usepackage{hyperref}
\usepackage{listings}
\usepackage{xcolor}
\usepackage{geometry}
\usepackage{booktabs}
\usepackage{longtable}
\usepackage{array}

% Page geometry
\geometry{margin=1in}

% Hyperref setup
\hypersetup{
	colorlinks=true,
	linkcolor=blue,
	filecolor=magenta,
	urlcolor=cyan,
}

% Listings setup for code blocks
\lstset{
	basicstyle=\ttfamily\small,
	breaklines=true,
	breakatwhitespace=true,
	frame=single,
	numbers=left,
	numberstyle=\tiny\color{gray},
	keywordstyle=\color{blue},
	commentstyle=\color{green!60!black},
	stringstyle=\color{red},
	showstringspaces=false,
	tabsize=2,
	captionpos=b,
	backgroundcolor=\color{gray!10}
}

% Define languages
\lstdefinelanguage{JavaScript}{
	keywords={break, case, catch, continue, debugger, default, delete, do, else, finally, for, function, if, in, instanceof, new, return, switch, this, throw, try, typeof, var, void, while, with, const, let, class, export, extends, import, super, true, false, null},
	morecomment=[l]{//},
	morecomment=[s]{/*}{*/},
	morestring=[b]',
	morestring=[b]",
	sensitive=true
}

\lstdefinelanguage{dockerfile}{
	keywords={FROM, RUN, CMD, LABEL, MAINTAINER, EXPOSE, ENV, ADD, COPY, ENTRYPOINT, VOLUME, USER, WORKDIR, ARG, ONBUILD, STOPSIGNAL, HEALTHCHECK, SHELL},
	sensitive=true,
	morecomment=[l]{\#},
	morestring=[b]"
}

\lstdefinelanguage{yaml}{
	keywords={true,false,null,y,n},
	sensitive=false,
	morecomment=[l]{\#},
	morestring=[b]',
	morestring=[b]"
}

\title{\textbf{SCD Project 25 - Docker Deployment Report}}
\author{\textbf{Student Name:} [Your Name]}
\date{\textbf{Date:} December 7, 2025\\[0.5em]\textbf{Course:} [Your Course Name]}

\begin{document}
	
	\maketitle
	\thispagestyle{empty}
	\newpage
	
	\tableofcontents
	\newpage
	
	\section{Part 3: Building Features into a Provided Project}
	
	\subsection{Step 1: Clone the Repository}
	
	\subsubsection{Commands Executed:}
	
	\begin{lstlisting}[language=bash]
		# Create working directory
		mkdir -p ~/scd-project
		cd ~/scd-project
		
		# Clone the repository
		git clone https://github.com/LaibaImran1500/SCDProject25.git
		
		# Navigate to the project
		cd SCDProject25
		
		# View project structure
		ls -la
	\end{lstlisting}
	
	\textbf{SCREENSHOT 1: Repository Cloned}
	
	\begin{figure}[h]
		\centering
		\includegraphics[width=0.8\textwidth]{image.png}
		\caption{Repository Cloned}
	\end{figure}
	
	\subsection{Step 2: Examine Project Structure}
	
	\begin{lstlisting}[language=bash]
		# View project structure
		ls -la
		
		# View all files recursively
		find . -type f -name "*.js" -o -name "*.json" | head -20
	\end{lstlisting}
	
	\textbf{Project Structure:}
	\begin{lstlisting}
		SCDProject25/
		├── main.js              # Main application entry point
		├── data/
		│   └── vault.json       # In-memory database (JSON file)
		├── db/
		│   ├── index.js         # Database operations (CRUD)
		│   ├── file.js          # File read/write operations
		│   └── record.js        # Record validation and ID generation
		└── events/
		├── index.js         # Event emitter
		└── logger.js        # Event logging
	\end{lstlisting}
	
	\textbf{Application Overview:}
	\begin{itemize}
		\item This is a NodeVault application - a CLI-based CRUD application
		\item Uses an in-memory JSON file database (\texttt{data/vault.json})
		\item Has event-driven logging for record operations
		\item Current menu options: Add, List, Update, Delete, Exit
	\end{itemize}
	
	\subsection{Step 3: Run the Application Locally}
	
	\begin{lstlisting}[language=bash]
		# Run the application
		node main.js
	\end{lstlisting}
	
	\textbf{Current Menu:}
	\begin{lstlisting}
		===== NodeVault =====
		1. Add Record
		2. List Records
		3. Update Record
		4. Delete Record
		5. Exit
		=====================
	\end{lstlisting}
	
	\textbf{SCREENSHOT 2: Application Running Locally}
	
	\begin{figure}[h]
		\centering
		\includegraphics[width=0.8\textwidth]{image-1.png}
		\caption{Application Running Locally}
	\end{figure}
	
	\subsection{Step 4: Create a Feature Branch}
	
	Before making any modifications, we create a new branch from the main branch. All changes will be made in this feature branch.
	
	\subsubsection{Commands Executed:}
	
	\begin{lstlisting}[language=bash]
		# Navigate to project directory
		cd ~/scd-project/SCDProject25
		
		# Check current branch
		git branch
		
		# Create and switch to feature branch
		git checkout -b feature/enhancements
		
		# Verify branch switch
		git branch
	\end{lstlisting}
	
	\textbf{SCREENSHOT 3: Feature Branch Created}
	
	\begin{figure}[h]
		\centering
		\includegraphics[width=0.8\textwidth]{image-2.png}
		\caption{Feature Branch Created}
	\end{figure}
	
	\subsection{Git Versioning Strategy}
	
	For every major change, we will create a version tag:
	
	\begin{table}[h]
		\centering
		\begin{tabular}{|l|l|l|}
			\hline
			\textbf{Feature} & \textbf{Version Tag} & \textbf{Command} \\
			\hline
			Search Functionality & v1.0 & \texttt{git tag -a v1.0 -m "Added search functionality"} \\
			\hline
			Sorting Capability & v1.1 & \texttt{git tag -a v1.1 -m "Added sorting capability"} \\
			\hline
			Export to Text File & v1.2 & \texttt{git tag -a v1.2 -m "Added export functionality"} \\
			\hline
			Automatic Backup & v1.3 & \texttt{git tag -a v1.3 -m "Added automatic backup"} \\
			\hline
			Data Statistics & v1.4 & \texttt{git tag -a v1.4 -m "Added vault statistics"} \\
			\hline
			MongoDB Setup & v1.5 & \texttt{git tag -a v1.5 -m "MongoDB integration"} \\
			\hline
			Env File Setup & v2.0 & \texttt{git tag -a v2.0 -m "Environment variables setup"} \\
			\hline
		\end{tabular}
		\caption{Git Versioning Strategy}
	\end{table}
	
	\section{Feature Implementations}
	
	\subsection{Feature 1: Search Functionality}
	
	\textbf{Description:} Allows users to search for existing records by name or ID (case-insensitive).
	
	\textbf{Implementation:}
	
	\begin{lstlisting}[language=JavaScript]
		// Function to search records (case-insensitive)
		function searchRecords(keyword) {
			const records = db.listRecords();
			const searchTerm = keyword.toLowerCase();
			
			return records.filter(r => 
			r.name.toLowerCase().includes(searchTerm) || 
			r.id.toString().includes(searchTerm) ||
			(r.value && r.value.toLowerCase().includes(searchTerm))
			);
		}
	\end{lstlisting}
	
	\textbf{Menu Option Added:} Option 5 - ``Search Records''
	
	\textbf{SCREENSHOT 4: Search Functionality}
	
	\begin{figure}[h]
		\centering
		\includegraphics[width=0.8\textwidth]{image-4.png}
		\caption{Search Functionality}
	\end{figure}
	
	\subsection{Feature 2: Sorting Capability}
	
	\textbf{Description:} Allows users to sort records by Name or Creation Date in Ascending or Descending order.
	
	\textbf{Implementation:}
	
	\begin{lstlisting}[language=JavaScript]
		// Function to sort records
		function sortRecords(field, order) {
			const records = [...db.listRecords()]; // Clone to avoid modifying original
			
			records.sort((a, b) => {
				let valA, valB;
				
				if (field === 'name') {
					valA = a.name.toLowerCase();
					valB = b.name.toLowerCase();
				} else if (field === 'date' || field === 'id') {
					valA = a.id; // ID is timestamp-based
					valB = b.id;
				}
				
				if (order === 'asc') {
					return valA > valB ? 1 : valA < valB ? -1 : 0;
				} else {
					return valA < valB ? 1 : valA > valB ? -1 : 0;
				}
			});
			
			return records;
		}
	\end{lstlisting}
	
	\textbf{Menu Option Added:} Option 6 - ``Sort Records''
	
	\textbf{Expected Output:}
	\begin{lstlisting}
		Choose field to sort by: Name
		Choose order: Ascending
		Sorted Records:
		1. ID: 104 | Name: Adeel
		2. ID: 110 | Name: Bilal
		3. ID: 108 | Name: Zain
	\end{lstlisting}
	
	\textbf{SCREENSHOT 5: Sorting Capability}
	
	\begin{figure}[h]
		\centering
		\includegraphics[width=0.8\textwidth]{image-3.png}
		\caption{Sorting Capability}
	\end{figure}
	
	\subsection{Feature 3: Export Vault Data to Text File}
	
	\textbf{Description:} Exports all records to a human-readable export.txt file.
	
	\textbf{Implementation:}
	
	\begin{lstlisting}[language=JavaScript]
		// Function to export data to text file
		function exportData() {
			const records = db.listRecords();
			const now = new Date();
			const dateStr = now.toISOString().replace('T', ' ').slice(0, 19);
			
			let content = `========================================\n`;
			content += `        NODEVAULT DATA EXPORT\n`;
			content += `========================================\n\n`;
			content += `Export Date/Time: ${dateStr}\n`;
			content += `Total Records: ${records.length}\n`;
			content += `File Name: export.txt\n`;
			// ... rest of formatting
			
			const exportPath = path.join(__dirname, 'export.txt');
			fs.writeFileSync(exportPath, content);
			return exportPath;
		}
	\end{lstlisting}
	
	\textbf{Menu Option Added:} Option 7 - ``Export Data''
	
	\textbf{SCREENSHOT 6: Export Functionality}
	
	\begin{figure}[h]
		\centering
		\includegraphics[width=0.8\textwidth]{image-5.png}
		\caption{Export Functionality - Part 1}
	\end{figure}
	
	\begin{figure}[h]
		\centering
		\includegraphics[width=0.8\textwidth]{image-6.png}
		\caption{Export Functionality - Part 2}
	\end{figure}
	
	\subsection{Feature 4: Automatic Backup System}
	
	\textbf{Description:} Automatically creates a backup whenever a record is added or deleted.
	
	\textbf{Implementation:}
	
	\begin{lstlisting}[language=JavaScript]
		// Backup directory
		const backupDir = path.join(__dirname, 'backups');
		if (!fs.existsSync(backupDir)) fs.mkdirSync(backupDir);
		
		// Function to create automatic backup
		function createBackup() {
			const now = new Date();
			const timestamp = now.toISOString().replace(/[:.]/g, '-').slice(0, 19);
			const backupFileName = `backup_${timestamp}.json`;
			const backupPath = path.join(backupDir, backupFileName);
			
			const records = db.listRecords();
			fs.writeFileSync(backupPath, JSON.stringify(records, null, 2));
			console.log(`Backup created: ${backupFileName}`);
		}
	\end{lstlisting}
	
	\textbf{Backup Location:} \texttt{/backups/backup\_YYYY-MM-DD\_HH-MM-SS.json}
	
	\textbf{SCREENSHOT 7: Automatic Backup}
	
	\begin{figure}[h]
		\centering
		\includegraphics[width=0.8\textwidth]{image-7.png}
		\caption{Automatic Backup - Part 1}
	\end{figure}
	
	\begin{figure}[h]
		\centering
		\includegraphics[width=0.8\textwidth]{image-8.png}
		\caption{Automatic Backup - Part 2}
	\end{figure}
	
	\subsection{Feature 5: Display Data Statistics}
	
	\textbf{Description:} Displays useful statistics about the vault data.
	
	\textbf{Implementation:}
	
	\begin{lstlisting}[language=JavaScript]
		// Function to display vault statistics
		function getVaultStatistics() {
			const records = db.listRecords();
			const vaultPath = path.join(__dirname, 'data', 'vault.json');
			
			const stats = {
				totalRecords: records.length,
				lastModified: 'N/A',
				longestName: 'N/A',
				longestNameLength: 0,
				earliestRecord: 'N/A',
				latestRecord: 'N/A'
			};
			
			// Get file modification time
			if (fs.existsSync(vaultPath)) {
				const fileStat = fs.statSync(vaultPath);
				stats.lastModified = fileStat.mtime.toISOString().replace('T', ' ').slice(0, 19);
			}
			
			// ... calculate other statistics
			
			return stats;
		}
	\end{lstlisting}
	
	\textbf{Menu Option Added:} Option 8 - ``View Vault Statistics''
	
	\textbf{Expected Output:}
	\begin{lstlisting}
		Vault Statistics:
		--------------------------
		Total Records: 5
		Last Modified: 2025-11-04 15:20:32
		Longest Name: Muhammad Abdullah (18 characters)
		Earliest Record: 2025-09-12
		Latest Record: 2025-11-02
		--------------------------
	\end{lstlisting}
	
	\textbf{SCREENSHOT 8: Vault Statistics}
	
	\begin{figure}[h]
		\centering
		\includegraphics[width=0.8\textwidth]{image-9.png}
		\caption{Vault Statistics}
	\end{figure}
	
	\subsection{Updated Menu Structure}
	
	\begin{lstlisting}
		===== NodeVault =====
		1. Add Record
		2. List Records
		3. Update Record
		4. Delete Record
		5. Search Records      (NEW)
		6. Sort Records        (NEW)
		7. Export Data         (NEW)
		8. View Vault Statistics (NEW)
		9. Exit
		=====================
	\end{lstlisting}
	
	\subsection{Git Commit for Features 1-5}
	
	\begin{lstlisting}[language=bash]
		# Stage all changes
		git add .
		
		# Commit with message
		git commit -m "Added search, sort, export, backup, and statistics features"
		
		# Create version tag
		git tag -a v3.0 -m "Version 1.0: Core feature enhancements"
	\end{lstlisting}
	
	\subsection{Step 5: Merge Feature Branch into Main}
	
	\begin{lstlisting}[language=bash]
		# Switch to main/master branch
		git checkout master
		
		# Merge feature branch
		git merge feature/enhancements
		
		# Push to remote (if applicable)
		git push origin master
		
		# Push tags
		git push --tags
	\end{lstlisting}
	
	\textbf{SCREENSHOT 14: Merge to Main}
	
	\begin{figure}[h]
		\centering
		\includegraphics[width=0.8\textwidth]{image-10.png}
		\caption{Merge to Main}
	\end{figure}
	
	\section{Part 4: Containerize the Application}
	
	\subsection{Step 2: Create Dockerfile}
	
	\textbf{Dockerfile:}
	
	\begin{lstlisting}[language=dockerfile]
		# Use Node.js 18 Alpine as base image
		FROM node:18-alpine
		
		# Set working directory
		WORKDIR /app
		
		# Copy package files
		COPY package*.json ./
		
		# Install dependencies
		RUN npm install --production
		
		# Copy application source code
		COPY . .
		
		# Create backups directory
		RUN mkdir -p /app/backups
		
		# Set environment to production
		ENV NODE_ENV=production
		
		# Expose port (if needed for future HTTP API)
		EXPOSE 3000
		
		# Command to run the application
		CMD ["node", "main.js"]
	\end{lstlisting}
	
	\textbf{.dockerignore:}
	
	\begin{lstlisting}
		node_modules
		npm-debug.log
		.git
		.gitignore
		.env
		*.md
		backups/*
		export.txt
		data/vault.json
	\end{lstlisting}
	
	\subsection{Step 3: Commit Dockerfile}
	
	\begin{lstlisting}[language=bash]
		# Stage changes
		git add .
		
		# Commit
		git commit -m "Added Dockerfile for containerization"
		
		# Create version tag
		git tag -a v3.0 -m "Version 3.0: Docker containerization"
	\end{lstlisting}
	
	\subsection{Step 4: Build Docker Image}
	
	\begin{lstlisting}[language=bash]
		# Build the Docker image
		docker build -t nodevault:v1 .
		
		# Verify image was created
		docker images | grep nodevault
	\end{lstlisting}
	
	\textbf{SCREENSHOT 17: Docker Build Process}
	
	\begin{figure}[h]
		\centering
		\includegraphics[width=0.8\textwidth]{image-11.png}
		\caption{Docker Build Process}
	\end{figure}
	
	\subsection{Step 5: Create Docker Network}
	
	\begin{lstlisting}[language=bash]
		# Create a network for the containers
		docker network create nodevault-network
	\end{lstlisting}
	
	\subsection{Step 6: Run MongoDB Container}
	
	\begin{lstlisting}[language=bash]
		# Run MongoDB container (if not already running)
		docker run -d \
		--name mongodb \
		--network nodevault-network \
		-p 27017:27017 \
		mongo:latest
	\end{lstlisting}
	
	\subsection{Step 7: Run NodeVault Container}
	
	\begin{lstlisting}[language=bash]
		# Run the NodeVault container
		docker run -it \
		--name nodevault-app \
		--network nodevault-network \
		-e MONGODB_URI=mongodb://mongodb:27017/nodevault \
		nodevault:v1
	\end{lstlisting}
	
	\textbf{SCREENSHOT 18: Container Running}
	
	\begin{figure}[h]
		\centering
		\includegraphics[width=0.8\textwidth]{image-12.png}
		\caption{Container Running}
	\end{figure}
	
	\subsection{Step 10: Publish to Docker Hub}
	
	\begin{lstlisting}[language=bash]
		# Login to Docker Hub
		docker login
		
		# Tag the image for Docker Hub
		# Replace YOUR_DOCKERHUB_USERNAME with your actual username
		docker tag nodevault:v1 zainalik157/nodevault:v1
		docker tag nodevault:v1 zainalik157/nodevault:latest
		
		# Push to Docker Hub
		docker push zainalik157/nodevault:v1
		docker push zainalik157/nodevault:latest
	\end{lstlisting}
	
	\textbf{SCREENSHOT 21: Docker Push}
	
	\begin{figure}[h]
		\centering
		\includegraphics[width=0.8\textwidth]{image-13.png}
		\caption{Docker Push}
	\end{figure}
	
	\textbf{Docker Hub URL:} \url{https://hub.docker.com/r/zainalik157/nodevault}
	
	\begin{figure}[h]
		\centering
		\includegraphics[width=0.8\textwidth]{image-14.png}
		\caption{Docker Hub Repository}
	\end{figure}
	
	\subsection{Step 11: Commit and Merge}
	
	\begin{lstlisting}[language=bash]
		# Commit any remaining changes
		git add .
		git commit -m "Finalized Docker containerization"
		
		# Switch to master and merge
		git checkout master
		git merge feature/containerization
		
		# Push tags
		git push --tags
	\end{lstlisting}
	
	\begin{figure}[h]
		\centering
		\includegraphics[width=0.8\textwidth]{image-15.png}
		\caption{Commit and Merge}
	\end{figure}
	
	\subsection{Summary - Part 4}
	
	\begin{table}[h]
		\centering
		\begin{tabular}{|l|c|}
			\hline
			\textbf{Task} & \textbf{Status} \\
			\hline
			Created containerization branch & \checkmark \\
			\hline
			Created Dockerfile & \checkmark \\
			\hline
			Created .dockerignore & \checkmark \\
			\hline
			Built Docker image & \checkmark \\
			\hline
			Tested with MongoDB container & \checkmark \\
			\hline
			Documented container logs & \checkmark \\
			\hline
			Documented container processes & \checkmark \\
			\hline
			Published to Docker Hub & \checkmark \\
			\hline
		\end{tabular}
		\caption{Part 4 Summary}
	\end{table}
	
	
	\section{Part 5: Deploy Containers Manually}
	
	\subsection{Overview}
	
	In this section, we deploy the containers manually using Docker CLI commands only (no YAML files). We will:
	\begin{enumerate}
		\item Create a private Docker network
		\item Attach volumes for persistent MongoDB data
		\item Configure ports and environment variables
		\item Demonstrate data persistence
	\end{enumerate}
	
	\subsection{Step 1: Clean Up Existing Containers}
	
	First, stop and remove any existing containers:
	
	\begin{lstlisting}[language=bash]
		# Stop and remove existing containers
		docker stop nodevault-app mongodb 2>/dev/null
		docker rm nodevault-app mongodb 2>/dev/null
		
		# Remove existing network
		docker network rm nodevault-network 2>/dev/null
		
		# Verify cleanup
		docker ps -a
	\end{lstlisting}
	
	\textbf{SCREENSHOT 22: Cleanup}
	
	\begin{figure}[h]
		\centering
		\includegraphics[width=0.8\textwidth]{image-16.png}
		\caption{Cleanup}
	\end{figure}
	
	\subsection{Step 2: Create Private Docker Network}
	
	\begin{lstlisting}[language=bash]
		# Create a private bridge network
		docker network create --driver bridge --internal nodevault-private-network
		
		# Verify network creation
		docker network ls
		
		# Inspect network details
		docker network inspect nodevault-private-network
	\end{lstlisting}
	
	\textbf{Explanation:}
	\begin{itemize}
		\item \texttt{--driver bridge}: Creates a bridge network for container communication
		\item \texttt{--internal}: Makes the network private (no external access)
		\item Containers on this network can communicate with each other but are isolated from the host network
	\end{itemize}
	
	\textbf{SCREENSHOT 23: Private Network Created}
	
	\begin{figure}[h]
		\centering
		\includegraphics[width=0.8\textwidth]{image-18.png}
		\caption{Private Network Created}
	\end{figure}
	
	\subsection{Step 3: Create Docker Volume for MongoDB Persistence}
	
	\begin{lstlisting}[language=bash]
		# Create a named volume for MongoDB data
		docker volume create mongodb-data
		
		# Verify volume creation
		docker volume ls
		
		# Inspect volume
		docker volume inspect mongodb-data
	\end{lstlisting}
	
	\textbf{SCREENSHOT 24: Volume Created}
	
	\begin{figure}[h]
		\centering
		\includegraphics[width=0.8\textwidth]{image-17.png}
		\caption{Volume Created}
	\end{figure}
	
	\subsection{Step 4: Run MongoDB Container with Volume}
	
	\begin{lstlisting}[language=bash]
		# Run MongoDB with volume attached
		docker run -d \
		--name mongodb \
		--network nodevault-private-network \
		-v mongodb-data:/data/db \
		-e MONGO_INITDB_DATABASE=nodevault \
		mongo:latest
		
		# Verify MongoDB is running
		docker ps
		
		# Check MongoDB logs
		docker logs mongodb
	\end{lstlisting}
	
	\textbf{Command Breakdown:}
	\begin{itemize}
		\item \texttt{-d}: Run in detached mode (background)
		\item \texttt{--name mongodb}: Container name
		\item \texttt{--network nodevault-private-network}: Connect to private network
		\item \texttt{-v mongodb-data:/data/db}: Mount volume for data persistence
		\item \texttt{-e MONGO\_INITDB\_DATABASE=nodevault}: Set initial database name
	\end{itemize}
	
	\textbf{SCREENSHOT 25: MongoDB Container Running}
	
	\begin{figure}[h]
		\centering
		\includegraphics[width=0.8\textwidth]{image-19.png}
		\caption{MongoDB Container Running}
	\end{figure}
	
	\subsection{Step 5: Run NodeVault Backend Container}
	
	\begin{lstlisting}[language=bash]
		# Run NodeVault backend container
		docker run -it \
		--name nodevault-app \
		--network nodevault-private-network \
		-e MONGODB_URI=mongodb://mongodb:27017/nodevault \
		-e NODE_ENV=production \
		nodevault:v1
		
		# For detached mode (background):
		docker run -d \
		--name nodevault-app \
		--network nodevault-private-network \
		-e MONGODB_URI=mongodb://mongodb:27017/nodevault \
		-e NODE_ENV=production \
		nodevault:v1
	\end{lstlisting}
	
	\textbf{Command Breakdown:}
	\begin{itemize}
		\item \texttt{-it}: Interactive mode with terminal
		\item \texttt{--network nodevault-private-network}: Same network as MongoDB
		\item \texttt{-e MONGODB\_URI=...}: Environment variable for database connection
		\item \texttt{-e NODE\_ENV=production}: Set production environment
	\end{itemize}
	
	\textbf{SCREENSHOT 26: NodeVault Container Running}
	
	\begin{figure}[h]
		\centering
		\includegraphics[width=0.8\textwidth]{image-20.png}
		\caption{NodeVault Container Running}
	\end{figure}
	
	\subsection{Step 6: Verify Network Isolation (Proof of Private Network)}
	
	\begin{lstlisting}[language=bash]
		# Check that containers are on the private network
		docker network inspect nodevault-private-network
		
		# Try to access MongoDB from host (should fail with internal network)
		curl http://localhost:27017 2>&1 || echo "Cannot access - Network is private!"
		
		# Verify containers can communicate internally
		docker exec nodevault-app ping -c 2 mongodb
	\end{lstlisting}
	
	\textbf{SCREENSHOT 27: Network Isolation Proof}
	
	\begin{figure}[h]
		\centering
		\includegraphics[width=0.8\textwidth]{image-21.png}
		\caption{Network Isolation Proof}
	\end{figure}
	
	\subsection{Step 7.1: Add Data to the Application}
	
	\begin{lstlisting}[language=bash]
		# Run the app and add some records
		docker run -it \
		--name nodevault-app \
		--network nodevault-private-network \
		-e MONGODB_URI=mongodb://mongodb:27017/nodevault \
		nodevault:v1
	\end{lstlisting}
	
	Add 2-3 records using the menu (option 1).
	
	\subsection{Step 7.2: Destroy and Recreate Containers}
	
	\begin{lstlisting}[language=bash]
		# Stop and remove the app container
		docker stop nodevault-app
		docker rm nodevault-app
		
		# Stop and remove MongoDB container
		docker stop mongodb
		docker rm mongodb
		
		# Verify containers are removed
		docker ps -a
	\end{lstlisting}
	
	\textbf{SCREENSHOT 29: Containers Destroyed}
	
	\begin{figure}[h]
		\centering
		\includegraphics[width=0.8\textwidth]{image-22.png}
		\caption{Containers Destroyed}
	\end{figure}
	
	\subsection{Step 7.3: Relaunch Containers}
	
	\begin{lstlisting}[language=bash]
		# Relaunch MongoDB with same volume
		docker run -d \
		--name mongodb \
		--network nodevault-private-network \
		-v mongodb-data:/data/db \
		mongo:latest
		
		# Wait for MongoDB to start
		sleep 5
		
		# Relaunch NodeVault
		docker run -it \
		--name nodevault-app \
		--network nodevault-private-network \
		-e MONGODB_URI=mongodb://mongodb:27017/nodevault \
		nodevault:v1
	\end{lstlisting}
	
	List records (option 2) - \textbf{Data should still be there!}
	
	\textbf{SCREENSHOT 30: Data Persistence Verified}
	
	The records still exist after container restart.
	
	\begin{figure}[h]
		\centering
		\includegraphics[width=0.8\textwidth]{image-23.png}
		\caption{Data Persistence Verified}
	\end{figure}
	
	\subsection{Complete List of Docker Commands Used}
	
	\begin{lstlisting}[language=bash]
		# Network Commands
		docker network create --driver bridge --internal nodevault-private-network
		docker network ls
		docker network inspect nodevault-private-network
		docker network rm nodevault-private-network
		
		# Volume Commands
		docker volume create mongodb-data
		docker volume ls
		docker volume inspect mongodb-data
		
		# MongoDB Container Commands
		docker run -d \
		--name mongodb \
		--network nodevault-private-network \
		-v mongodb-data:/data/db \
		-e MONGO_INITDB_DATABASE=nodevault \
		mongo:latest
		
		# NodeVault Container Commands
		docker run -it \
		--name nodevault-app \
		--network nodevault-private-network \
		-e MONGODB_URI=mongodb://mongodb:27017/nodevault \
		-e NODE_ENV=production \
		nodevault:v1
		
		# Management Commands
		docker ps
		docker ps -a
		docker logs <container_name>
		docker stop <container_name>
		docker rm <container_name>
		docker exec <container_name> <command>
	\end{lstlisting}
	
	\subsection{Difficulties in Manual Container Setup}
	
	\begin{table}[h]
		\centering
		\begin{tabular}{|p{4cm}|p{10cm}|}
			\hline
			\textbf{Challenge} & \textbf{Description} \\
			\hline
			Network Configuration & Manually creating and managing networks requires understanding of Docker networking concepts. Easy to misconfigure. \\
			\hline
			Volume Management & Must remember to attach volumes correctly every time. Missing \texttt{-v} flag loses all data. \\
			\hline
			Environment Variables & Must pass all env vars via \texttt{-e} flags. Easy to forget or mistype. \\
			\hline
			Container Dependencies & Must start containers in correct order (MongoDB before app). No automatic dependency management. \\
			\hline
			Command Length & Commands become very long with all options. Error-prone when typing manually. \\
			\hline
			Reproducibility & Hard to reproduce exact same setup. Must remember all flags and options. \\
			\hline
			Port Conflicts & Must manually track which ports are in use. \\
			\hline
			No Health Checks & No automatic restart if container fails. \\
			\hline
		\end{tabular}
		\caption{Difficulties in Manual Container Setup}
	\end{table}
	
	\subsection{Time and Effort Analysis}
	
	\begin{table}[h]
		\centering
		\begin{tabular}{|l|l|}
			\hline
			\textbf{Task} & \textbf{Estimated Time} \\
			\hline
			Understanding Docker networking & 30-60 minutes \\
			\hline
			Creating and testing network & 15-20 minutes \\
			\hline
			Setting up volumes & 10-15 minutes \\
			\hline
			Configuring MongoDB container & 15-20 minutes \\
			\hline
			Configuring NodeVault container & 15-20 minutes \\
			\hline
			Testing and debugging & 30-45 minutes \\
			\hline
			Documenting commands & 20-30 minutes \\
			\hline
			\textbf{Total} & \textbf{2-3 hours} \\
			\hline
		\end{tabular}
		\caption{Time and Effort Analysis}
	\end{table}
	
	\textbf{Conclusion:} Manual container deployment is time-consuming, error-prone, and difficult to maintain. This highlights the need for Docker Compose (Part 6) to simplify the process.
	
	\section{Part 6: Simplifying with Docker Compose}
	
	\subsection{Overview}
	
	Docker Compose simplifies multi-container deployment by defining all services, networks, and volumes in a single YAML file. This eliminates the need for multiple long Docker CLI commands.
	
	\subsection{Step 1: Create docker-compose.yml}
	
	\begin{lstlisting}[language=yaml]
		version: '3.8'
		
		services:
		# MongoDB Database Service
		mongodb:
		image: mongo:latest
		container_name: nodevault-mongodb
		restart: unless-stopped
		environment:
		- MONGO_INITDB_DATABASE=nodevault
		volumes:
		- mongodb-data:/data/db
		networks:
		- nodevault-network
		healthcheck:
		test: echo 'db.runCommand("ping").ok' | mongosh localhost:27017/nodevault --quiet
		interval: 10s
		timeout: 5s
		retries: 5
		
		# NodeVault Backend Service
		backend:
		image: nodevault:v1
		container_name: nodevault-backend
		restart: unless-stopped
		depends_on:
		mongodb:
		condition: service_healthy
		environment:
		- MONGODB_URI=mongodb://mongodb:27017/nodevault
		- NODE_ENV=production
		env_file:
		- .env
		networks:
		- nodevault-network
		stdin_open: true
		tty: true
		
		# Custom Bridge Network
		networks:
		nodevault-network:
		driver: bridge
		name: nodevault-compose-network
		
		# Persistent Volume for MongoDB
		volumes:
		mongodb-data:
		driver: local
		name: nodevault-mongodb-data
	\end{lstlisting}
	
	\subsection{Step 2: Update .env File}
	
	\begin{lstlisting}
		# MongoDB Connection String
		MONGODB_URI=mongodb://mongodb:27017/nodevault
		
		# Node Environment
		NODE_ENV=production
	\end{lstlisting}
	
	\subsection{Step 3: Stop Existing Containers}
	
	\begin{lstlisting}[language=bash]
		# Stop and remove manually created containers
		docker stop nodevault-app mongodb 2>/dev/null
		docker rm nodevault-app mongodb 2>/dev/null
		
		# Remove old network
		docker network rm nodevault-private-network 2>/dev/null
	\end{lstlisting}
	
	\subsection{Step 4: Start Services with Docker Compose}
	
	\begin{lstlisting}[language=bash]
		cd ~/Desktop/SCD\ Project/SCDProject25
		
		# Start all services
		docker-compose up -d
		
		# Or with build flag (if image needs rebuilding)
		docker-compose up -d --build
	\end{lstlisting}
	
	\textbf{SCREENSHOT 32: Docker Compose Up}
	
	\begin{figure}[h]
		\centering
		\includegraphics[width=0.8\textwidth]{image-24.png}
		\caption{Docker Compose Up}
	\end{figure}
	
	\subsection{Step 5: Verify Services are Running}
	
	\begin{lstlisting}[language=bash]
		# Check running containers
		docker-compose ps
		
		# Or use docker ps
		docker ps
	\end{lstlisting}
	
	\textbf{SCREENSHOT 33: Services Running}
	
	\begin{figure}[h]
		\centering
		\includegraphics[width=0.8\textwidth]{image-25.png}
		\caption{Services Running}
	\end{figure}
	
	\subsection{Step 7: Test the Application}
	
	\begin{lstlisting}[language=bash]
		# Attach to the backend container
		docker attach nodevault-backend
		
		# Or run interactively
		docker-compose exec backend node main.js
	\end{lstlisting}
	
	\textbf{SCREENSHOT 35: Application Working}
	
	\begin{figure}[h]
		\centering
		\includegraphics[width=0.8\textwidth]{image-26.png}
		\caption{Application Working}
	\end{figure}
	
	\subsection{Step 8: Verify Network and Volumes}
	
	\begin{lstlisting}[language=bash]
		# Check network
		docker network ls | grep nodevault
		
		# Check volumes
		docker volume ls | grep nodevault
		
		# Inspect network
		docker network inspect nodevault-compose-network
	\end{lstlisting}
	
	\textbf{SCREENSHOT 36: Network and Volumes}
	
	\begin{figure}[h]
		\centering
		\includegraphics[width=0.8\textwidth]{image-27.png}
		\caption{Network and Volumes}
	\end{figure}
	
	\subsection{Step 9: Stop Services}
	
	\begin{lstlisting}[language=bash]
		# Stop all services
		docker-compose down
		
		# Stop and remove volumes (careful - deletes data!)
		docker-compose down -v
	\end{lstlisting}
	
	\subsection{Docker Compose vs Manual Deployment Comparison}
	
	\begin{table}[h]
		\centering
		\begin{tabular}{|l|l|l|}
			\hline
			\textbf{Aspect} & \textbf{Manual CLI} & \textbf{Docker Compose} \\
			\hline
			Commands needed & 10+ commands & 1 command \\
			\hline
			Configuration & Flags in command line & YAML file \\
			\hline
			Reproducibility & Hard to reproduce & Easy - just share YAML \\
			\hline
			Network setup & Manual creation & Automatic \\
			\hline
			Volume setup & Manual creation & Automatic \\
			\hline
			Dependencies & Manual ordering & \texttt{depends\_on} handles it \\
			\hline
			Environment vars & Multiple \texttt{-e} flags & \texttt{.env} file \\
			\hline
			Maintenance & Edit commands & Edit YAML file \\
			\hline
			Team collaboration & Share commands & Share docker-compose.yml \\
			\hline
		\end{tabular}
		\caption{Docker Compose vs Manual Deployment Comparison}
	\end{table}
	
	\subsection{Benefits of Docker Compose}
	
	\begin{enumerate}
		\item \textbf{Single Command Deployment:} \texttt{docker-compose up} starts everything
		\item \textbf{Declarative Configuration:} All settings in one readable YAML file
		\item \textbf{Automatic Networking:} Services can communicate by name
		\item \textbf{Volume Management:} Persistent storage defined in config
		\item \textbf{Environment Variables:} Loaded from \texttt{.env} file automatically
		\item \textbf{Health Checks:} Ensures dependencies are ready before starting
		\item \textbf{Easy Scaling:} Can scale services with \texttt{docker-compose up --scale}
		\item \textbf{Version Control:} YAML file can be committed to git
	\end{enumerate}
	
	\subsection{Summary - Part 6}
	
	\begin{table}[h]
		\centering
		\begin{tabular}{|l|c|}
			\hline
			\textbf{Task} & \textbf{Status} \\
			\hline
			Created docker-compose.yml & \checkmark \\
			\hline
			Defined backend service & \checkmark \\
			\hline
			Defined database service & \checkmark \\
			\hline
			Configured custom network & \checkmark \\
			\hline
			Configured volumes & \checkmark \\
			\hline
			Used .env file & \checkmark \\
			\hline
			Services up with one command & \checkmark \\
			\hline
		\end{tabular}
		\caption{Part 6 Summary}
	\end{table}
	
	
	\section{Part 7: Update Project Repo to include Docker Compose}
	
	\subsection{Step 1: Update docker-compose.yml to Build from Dockerfile}
	
	The docker-compose.yml is updated to build the image from the Dockerfile instead of using a pre-built image:
	
	\begin{lstlisting}[language=yaml]
		services:
		# MongoDB Database Service
		mongodb:
		image: mongo:latest
		container_name: nodevault-mongodb
		restart: unless-stopped
		environment:
		- MONGO_INITDB_DATABASE=nodevault
		volumes:
		- mongodb-data:/data/db
		networks:
		- nodevault-network
		healthcheck:
		test: echo 'db.runCommand("ping").ok' | mongosh localhost:27017/nodevault --quiet
		interval: 10s
		timeout: 5s
		retries: 5
		
		# NodeVault Backend Service
		backend:
		build:
		context: .
		dockerfile: Dockerfile
		image: nodevault:latest
		container_name: nodevault-backend
		restart: unless-stopped
		depends_on:
		mongodb:
		condition: service_healthy
		environment:
		- MONGODB_URI=mongodb://mongodb:27017/nodevault
		- NODE_ENV=production
		networks:
		- nodevault-network
		stdin_open: true
		tty: true
		
		# Custom Bridge Network
		networks:
		nodevault-network:
		driver: bridge
		name: nodevault-compose-network
		
		# Persistent Volume for MongoDB
		volumes:
		mongodb-data:
		driver: local
		name: nodevault-mongodb-data
	\end{lstlisting}
	
	\textbf{Key Change:} Added \texttt{build} section to backend service:
	\begin{lstlisting}[language=yaml]
		build:
		context: .
		dockerfile: Dockerfile
	\end{lstlisting}
	
	\subsection{Step 2: Clean Slate - Remove All Docker Images}
	
	\begin{lstlisting}[language=bash]
		# Stop all running containers
		docker compose down
		
		# Remove all unused containers, networks, images, and volumes
		docker system prune -a
		
		# Verify images are removed
		docker images
	\end{lstlisting}
	
	\textbf{WARNING:} \texttt{docker system prune -a} removes ALL unused Docker resources. Use with caution!
	
	\textbf{SCREENSHOT 37: Clean Slate}
	
	\begin{figure}[h]
		\centering
		\includegraphics[width=0.8\textwidth]{image-28.png}
		\caption{Clean Slate - Part 1}
	\end{figure}
	
	\begin{figure}[h]
		\centering
		\includegraphics[width=0.8\textwidth]{image-29.png}
		\caption{Clean Slate - Part 2}
	\end{figure}
	
	\subsection{Step 3: Build and Run with Docker Compose}
	
	\begin{lstlisting}[language=bash]
		cd ~/Desktop/SCD\ Project/SCDProject25
		
		# Build and start all services
		docker compose up --build
	\end{lstlisting}
	
	\textbf{SCREENSHOT 38: Docker Build Process}
	
	\begin{figure}[h]
		\centering
		\includegraphics[width=0.8\textwidth]{image-30.png}
		\caption{Docker Build Process}
	\end{figure}
	
	\textbf{SCREENSHOT 39: Services Running}
	
	\begin{figure}[h]
		\centering
		\includegraphics[width=0.8\textwidth]{image-31.png}
		\caption{Services Running}
	\end{figure}
	
	\subsection{Step 4: Verify Application is Working}
	
	\begin{lstlisting}[language=bash]
		# In another terminal, check running containers
		docker ps
		
		# Attach to backend to use the application
		docker attach nodevault-backend
	\end{lstlisting}
	
	\begin{figure}[h]
		\centering
		\includegraphics[width=0.8\textwidth]{image-32.png}
		\caption{Running Containers}
	\end{figure}
	
	\textbf{SCREENSHOT 40: Application Functioning}
	
	\begin{figure}[h]
		\centering
		\includegraphics[width=0.8\textwidth]{image-33.png}
		\caption{Application Functioning - Part 1}
	\end{figure}
	
	\begin{figure}[h]
		\centering
		\includegraphics[width=0.8\textwidth]{image-34.png}
		\caption{Application Functioning - Part 2}
	\end{figure}
	
	\subsection{Step 5: Create README.md}
	
	A README.md file has been created with:
	\begin{itemize}
		\item Project description
		\item Features list
		\item Prerequisites
		\item Quick start instructions
		\item Docker commands
		\item Project structure
		\item Menu options
	\end{itemize}
	
	\subsection{Step 6: Commit and Push to GitHub}
	
	\begin{lstlisting}[language=bash]
		cd ~/Desktop/SCD\ Project/SCDProject25
		
		# Check current branch
		git branch
		
		# Switch to master if needed
		git checkout master
		
		# Merge feature branch (if not already merged)
		git merge feature/containerization
		
		# Stage all changes
		git add .
		
		# Commit
		git commit -m "Added Docker Compose with build configuration and README"
		
		# Create final version tag
		git tag -a v4.0 -m "Version 4.0: Complete Docker Compose setup"
		
		# Push to GitHub
		git push origin master
		
		# Push tags
		git push --tags
	\end{lstlisting}
	
	\textbf{SCREENSHOT 41: Git Commit and Push}
	\begin{itemize}
		\item Take a screenshot showing the commit
		\item Take a screenshot showing the push to GitHub
	\end{itemize}
	
	\subsection{Step 7: Verify on GitHub}
	
	Visit your GitHub repository to verify:
	\begin{itemize}
		\item docker-compose.yml is present
		\item Dockerfile is present
		\item README.md is present
		\item .env.example is present (not .env - it should be in .gitignore)
	\end{itemize}
	
	\textbf{SCREENSHOT 42: GitHub Repository}
	
	\begin{figure}[h]
		\centering
		\includegraphics[width=0.8\textwidth]{1image.png}
		\caption{GitHub Repository}
	\end{figure}
	
	\subsection{Files Committed}
	
	\begin{table}[h]
		\centering
		\begin{tabular}{|l|l|}
			\hline
			\textbf{File} & \textbf{Description} \\
			\hline
			\texttt{docker-compose.yml} & Docker Compose configuration \\
			\hline
			\texttt{Dockerfile} & Docker image build instructions \\
			\hline
			\texttt{.dockerignore} & Files to exclude from Docker build \\
			\hline
			\texttt{README.md} & Project documentation \\
			\hline
			\texttt{.env.example} & Example environment variables \\
			\hline
			\texttt{.gitignore} & Git ignore rules \\
			\hline
			\texttt{main.js} & Main application with all features \\
			\hline
			\texttt{db/} & Database layer with MongoDB \\
			\hline
			\texttt{events/} & Event handling \\
			\hline
			\texttt{package.json} & Node.js dependencies \\
			\hline
		\end{tabular}
		\caption{Files Committed}
	\end{table}
	
	\subsection{Issues Encountered and Solutions}
	
	\begin{table}[h]
		\centering
		\begin{tabular}{|p{5cm}|p{9cm}|}
			\hline
			\textbf{Issue} & \textbf{Solution} \\
			\hline
			\texttt{docker-compose} command not found & Use \texttt{docker compose} (without hyphen) on newer Docker versions \\
			\hline
			Version attribute warning & Removed \texttt{version: '3.8'} as it's obsolete in newer Docker Compose \\
			\hline
			MongoDB health check & Added proper health check to ensure MongoDB is ready before backend starts \\
			\hline
			Interactive CLI in container & Added \texttt{stdin\_open: true} and \texttt{tty: true} for interactive mode \\
			\hline
		\end{tabular}
		\caption{Issues Encountered and Solutions}
	\end{table}
	
	\subsection{Summary - Part 7}
	
	\begin{table}[h]
		\centering
		\begin{tabular}{|l|c|}
			\hline
			\textbf{Task} & \textbf{Status} \\
			\hline
			Updated docker-compose.yml with build config & \checkmark \\
			\hline
			Cleaned Docker environment & \checkmark \\
			\hline
			Built images with docker compose up --build & \checkmark \\
			\hline
			Verified application working & \checkmark \\
			\hline
			Created README.md & \checkmark \\
			\hline
			Committed all changes & \checkmark \\
			\hline
			Pushed to GitHub & \checkmark \\
			\hline
		\end{tabular}
		\caption{Part 7 Summary}
	\end{table}
	
	\section{Final Project Summary}
	
	\subsection{All Parts Completed}
	
	\begin{table}[h]
		\centering
		\begin{tabular}{|l|l|c|}
			\hline
			\textbf{Part} & \textbf{Description} & \textbf{Status} \\
			\hline
			Part 3 & Feature Implementation (Search, Sort, Export, Backup, Stats, MongoDB) & \checkmark \\
			\hline
			Part 4 & Containerize Application & \checkmark \\
			\hline
			Part 5 & Manual Container Deployment & \checkmark \\
			\hline
			Part 6 & Docker Compose Setup & \checkmark \\
			\hline
			Part 7 & Update Repo with Docker Compose & \checkmark \\
			\hline
		\end{tabular}
		\caption{All Parts Completed}
	\end{table}
	
	\subsection{Technologies Used}
	
	\begin{itemize}
		\item Node.js 18
		\item MongoDB
		\item Docker
		\item Docker Compose
		\item Git/GitHub
	\end{itemize}
	
	\subsection{Key Learnings}
	
	\begin{enumerate}
		\item \textbf{Environment Consistency:} Docker ensures the same environment across development and production
		\item \textbf{Container Orchestration:} Docker Compose simplifies multi-container deployments
		\item \textbf{Data Persistence:} Volumes ensure data survives container restarts
		\item \textbf{Network Isolation:} Private networks secure container communication
		\item \textbf{Infrastructure as Code:} docker-compose.yml defines entire infrastructure
	\end{enumerate}
	
\end{document}
